\documentclass[a4paper,11pt]{report}
\usepackage[T1]{fontenc}
\usepackage[utf8]{inputenc}
\usepackage{lmodern}
\usepackage[francais]{babel}

\title{Rapport de projet - Vie dans un labyrinthe}
\author{BODINEAU Bastien, DURAN Alizée et NGATCHOU Junior}
\date{Novembre 2015 - Janvier 2016}

\begin{document}

\maketitle
\tableofcontents

\chapter{Modalités et présentation du sujet}
Dans cette première partie, nous allons expliquer le sujet ainsi que les modifications effectuées.
\section{Sujet du projet choisi}
Blabla.
\section{Modifications apportées au sujet}
Blabla.
\chapter{Organisation du travail}
Ici, nous allons détailler la répartition du travail entre les étudiants en attribuant chacun un module différent.
\section{Répartition des tâches}
Bla
\section{Squelette du programme}
Ensuite, nous allons donner le squelette du programme (correspondant au main).
\chapter{Analyse du problème}
Dans cette partie, nous allons indiquer les structures de données que nous avons établies ainsi que la génération du labyrinthe, qui est un élément important du projet.
\section{Structures de données}
Voici ici les différentes structures de données que chaque module possède.
\section{Génération du labyrinthe}
Puis, nous allons présenter la manière dont nous avons réalisé la gestion du labyrinthe.
\chapter{Codage du jeu}
Dans cette partie, nous allons expliciter les fonctions utiles à la programmation du jeu.
\chapter{Gestion des fichiers - Github}
Pour avoir accès à nos fichiers et les mettre en partage, nous avons utilisé le dépôt distant Github.
\chapter{Résultats}
Cette partie constitue les résultats que nous a donné le jeu lors de sa programmation.
\section{Implémentations}
Ici les implémentations.
\section{Fonctions qui fonctionnent}
Fonctions que le programme peut effectuer
\section{Ajout de compléments}
Si on a le temps, et si c'est réussi ou non, montrer tous les extras.
\section{Conclusion}
En prenant du recul, nous pouvons dire que ce projet nous a permis de créer un jeu avec tous les outils mis à disposition.
Problèmes rencontrés.
Améliorations.
\chapter{Annexes}
Si il y en a.
\end{document}
