\documentclass[a4paper,11pt]{report}
\usepackage[T1]{fontenc}
\usepackage[utf8]{inputenc}
\usepackage{lmodern}
\usepackage[francais]{babel}
\usepackage{array,multirow}
\usepackage{listings}
\usepackage{fancyhdr}
\usepackage{xcolor}

\setlength{\headheight}{14.998pt}
\pagestyle{fancy}
\fancyhead[L]{BODINEAU Bastien, DURAN Alizée et NGATCHOU Junior}
\fancyhead[R]{ Université du maine}

\newcolumntype{R}[1]{>{\raggedleft\arraybackslash }b{#1}}
\newcolumntype{L}[1]{>{\raggedright\arraybackslash }b{#1}}
\newcolumntype{C}[1]{>{\centering\arraybackslash }b{#1}}

\lstset{
  belowcaptionskip=1\baselineskip,
  breaklines=true,
  tabsize=4,
  language=C,
  showstringspaces=false,
  basicstyle=\footnotesize\ttfamily,
  keywordstyle=\bfseries\color{green!40!black},
  commentstyle=\itshape\color{purple!40!black},
  identifierstyle=\color{blue},
  stringstyle=\color{orange}
}

\title{Rapport de projet - Vie dans un labyrinthe}
\author{BODINEAU Bastien, DURAN Alizée et NGATCHOU Junior}
\date{Novembre 2015 - Janvier 2016}

\begin{document}

\maketitle
\tableofcontents

\chapter{Présentation du sujet}
Dans cette première partie, nous allons expliquer le sujet ainsi que les modifications effectuées.
\section{Sujet du projet choisi}
Pour l'UE de Conduite de projet, il nous faut réaliser un jeu sur terminal avec possibilité de le faire avec une interface graphique (SDL). Parmi plusieurs sujets, nous avons choisi "Vie dans un labyrinthe". 

Le but de ce jeu est de faire évoluer des insectes à l'intérieur d'un labyrinthe, en s'inspirant du jeu de la vie, un jeu que l'on a déjà réalisé au cours de l'UE de Algorithme et Programmation. Nous devons également informer l'utilisateur sur les statistiques de la population présente dans le labyrinthe, générer des déplacements semi-aléatoires et nourrir ces insectes.
\section{Modifications apportées au sujet}
Dans ce sujet, nous avons effectivement apporté des modifications afin de mieux étudier le sujet du jeu à réaliser. En premier lieu, nous avons préféré mettre des entités comme un joueur et un/plusieurs monstre(s). Les insectes peuvent tout à fait être des monstres que le joueur doit combattre.

Ensuite, le codage de la génération du labyrinthe, étant un point clé du jeu, sera explicité tout au long de ce rapport.
\chapter{Organisation du travail}
Ici, nous allons détailler la répartition du travail entre les étudiants en attribuant chacun un module différent.
\section{Répartition des tâches}
Afin de concevoir le programme du jeu, nous avons réparti les tâches en fonction de modules. En effet, trois modules principaux sont importants : le module du labyrinthe, le module du joueur et le module du monstre. Voici la répartition des tâches en fonction des modules et des étudiants :
\begin{table}[htbp]
  \center
  \caption{Répartition des tâches.}
  \begin{tabular}{|c|c c c|}
   \hline 
                               &Joueur    &Labyrinthe   &Monstre   \\
                       Alizée  & $\times$ &             &          \\
                       Bastien &          &$\times$     &          \\
                       Junior  &          &             &$\times$  \\
   \hline
  \end{tabular}
\end{table}
\section{Squelette du programme}
Ensuite, nous allons donner le squelette du programme :

\begin{lstlisting}[language=c]
int main(){
  //Mettre le code du main plus tard
}
\end{lstlisting}
\chapter{Analyse du problème}
Dans cette partie, nous allons indiquer les structures de données que nous avons établies ainsi que la génération du labyrinthe, qui est un élément important du projet.
\section{Structures de données}
Voici ici les différentes structures de données que chaque module possède.
\section{Génération du labyrinthe}
Puis, nous allons présenter la manière dont nous avons réalisé la gestion du labyrinthe.
\chapter{Codage du jeu}
Dans cette partie, nous allons expliciter les fonctions utiles à la programmation du jeu.
\chapter{Gestion des fichiers - Github}
Pour avoir accès à nos fichiers et les mettre en partage, nous avons utilisé le dépôt distant Github.
\chapter{Résultats}
Cette partie constitue les résultats que nous a donné le jeu lors de sa programmation.
\section{Implémentations}
Ici les implémentations.
\section{Fonctions qui fonctionnent}
Fonctions que le programme peut effectuer
\section{Ajout de compléments}
Si on a le temps, et si c'est réussi ou non, montrer tous les extras.
\section{Conclusion}
En prenant du recul, nous pouvons dire que ce projet nous a permis de créer un jeu avec tous les outils mis à disposition.
Problèmes rencontrés.
Améliorations.
\chapter{Annexes}
Si il y en a.
\end{document}
